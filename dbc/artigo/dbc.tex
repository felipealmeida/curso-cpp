%% -*- TeX-engine: xetex; mode: LaTeX; eval: (TeX-PDF-mode); coding: utf-8-unix; TeX-master: t; ispell-local-dictionary: "pt_BR"; -*-

\documentclass[12pt]{article}

\usepackage{sbc-template}

\usepackage{graphicx,url}

\usepackage{fontspec}
\usepackage{listings}

%\usepackage[brazil]{babel}   
%\usepackage[latin1]{inputenc}  

%\sloppy

\title{Programação por contrato}

\author{Felipe Magno de Almeida}

\address{Expertise Solutions}

\begin{document} 

\maketitle
     
\begin{resumo} 
  
\end{resumo}

\section{Programação imperativa}

A programação imperativa consiste da execução sequencial de sentenças
que fazem modificações no estado da aplicação. Essas sentenças são
como sentenças no modo imperativo do português, que expressa ordem,
comando. Esse é um paradigma intrinsicamente diferente da programação
funcional, por exemplo, que tem progressos através de avaliações
sucessivas de expressões.

\section{Programação por Contrato}



\section{Estado}

Para falar sobre sentenças e modificações de estados, precisamos definir 

É possível pensar, formalmente, em estado de um programa de computador
como um conjunto de pares ordenados. Aonde o primeiro elemento do par
identifica a variável de nosso programa, e.g. por um identificador, e
o segundo elemento do par o valor dessa variável naquele estado.

Por exemplo o programa abaixo chamado com $x=5$:

\begin{lstlisting}[mathescape=true]
int incr(int $x$)
{
  int $y$ = 10;
  // $S = \left\{{ \left< x,5 \right>, \left< y,10 \right> }\right\}$
  ++$x$;
  // $S = \left\{{ \left< x,6 \right>, \left< y,10 \right> }\right\}$
  return $x$;
}
\end{lstlisting}

Como um conjunto não possui relação de ordem entre seus elementos,
$\left\{{ \left< x,6 \right>, \left< y,10 \right> }\right\} = \left\{{
    \left< y,10 \right>, \left< x,6 \right> }\right\}$. É por isso que
a simples representação do estado como conjunto de valores e fazer a
relação entre variável e valor através de índice não seria suficiente,
como por exemplo $S = \left\{{5,10}\right\}$, já que
$\left\{{5,10}\right\} = \left\{{10,5}\right\}$. Também o uso de pares
ordenados demonstra uma relação binária entre o conjunto de variáveis
e o conjunto de valores que representam aquelas variáveis em
determinado momento.

O conceito de estado como um conjunto formaliza uma série de operações
essenciais como a escrita de predicados e etc. Porém, para facilitar,
este artigo usará a sintaxe $x = 5$ como equivalente a
$\left<x,5\right> \in S$, aonde $S$ é o conjunto do estado corrente.

\end{document}
